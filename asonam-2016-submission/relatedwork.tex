\section{Related Work}
Radio frequency and ultrasonic signals are used in conjunction
\cite{priyantha2000cricket} to estimate the location of mobile devices, with
the overhead of installing beacon devices throughout the building and listener
devices to user mobiles.  
%This method only works in indoor environments and
%requires pre-deployment equipment installation.  
%Relying only on WiFi signals
%though does not require any pre-deployment overhead %since WiFi access points
%already exist everywhere.  
Received signal strength indication (RSSI)
measurements between the cellular base stations and the user device
S\cite{liu1997hierarchical} are processed by algorithms combining an extended
Kalman filter, approximate pattern matching and velocity vectors to predict
future movements of users across cell boundaries.  It is suitable for only coarse-scale outdoor environments.
%and does not apply naturally to small-scale user mobility,
%e.g., indoor inter-room movements.

Indoor localization algorithms that use RSSI have been extensively investigated
\cite{kushki2007kernel,liu2005signal,
hatami2005comparative,paul2008wi,feng2012received} because there is no overhead
cost of additional hardware required before deployment.  In \cite{paul2008wi},
positioning and tracking of users depend on the locations of known WiFi access
points which may not be realistic in the general case.  Our work does not
require such prior knowledge, we use readings from any available access points
that are detected.  Some work \cite{bahl2000radar,ferris2006gaussian} focus on
the problems arising from the unpredictable effect of physical factors on the
received signals and use a propagation model for the relationship between
signal and location.  An alternative is the fingerprinting model
\cite{kaemarungsi2004modeling,li2006indoor, ma2008cluster} where offline
readings are recorded first and the online readings are then compared to them.
The $k$ nearest neighbor algorithm is used in \cite{li2006indoor,ma2008cluster}
to estimate the user location.  It is an efficient algorithm but suffers in
estimation accuracy.  The localization system in \cite{feng2012received} has
two phases: an offline phase in which a radio map is built by collecting
readings on a grid of reference points and decomposed into clusters; an online
phase in which signal readings are first identified to belong to certain
clusters and then more finely localized using compression sensing theory.  This
method has high accuracy but still requires an initial offline phase which is
not realistic in general, especially in outdoor environments.

In works such as \cite{roos2002probabilistic,smailagic2002location}, the indoor
area is divided into a grid and signal data is recorded from a set of fixed
known points which is used as training data, generating a probability
distribution of signal strengths given location values.  With new signal
observations, the posterior distribution is computed and the location with the
highest probability is chosen.  

The works of \cite{seshadri2005bayesian} and \cite{letchner2005large} both
model user locations as states of a dynamic system with the noisy observations
of RSSI data, and apply an implementation of Bayesian filters called particle
filters to probabilistically estimate the system state (i.e., user location).
An initial training phase builds a wireless sensor map of the environment
divided into cells by sampling at predefined points; this step is dependent on
the assumed WiFi sensor model, either signal propagation or fingerprinting
models.  Then location is estimated on a spatial connectivity graph; each
motion update step of the Bayesian filter moves the user along an edge of this
graph.  These probabilistic methods are powerful but suffers the drawback of
potential high computational complexity inherent in the particle filters method
whose worst-case complexity grows exponentially in dimensions of the state
space.  Moreover, all these above-mentioned approaches all require an initial
training phase, and therefore are not online and ad-hoc.

