\section{Related Work}
Radio frequency and ultrasonic signals are used in conjunction \cite{priyantha2000cricket} to estimate the location of mobile devices, with the overhead of installing beacon devices throughout the building and listener devices to user mobiles.  This method only works in indoor environments and requires pre-deployment equipment installation.  Relying only on WiFi signals though does not require any pre-deployment overhead since WiFi access points already exist everywhere.
Received signal strength indication (RSSI) measurements between the cellular base stations and the user device \cite{liu1997hierarchical} are processed by algorithms combining an extended Kalman filter, approximate pattern matching and velocity vectors to predict future movements of users across cell boundaries.  It is suitable for only outdoor environments and does not apply naturally to small-scale user mobility, e.g., indoor inter-room movements.
In works such as \cite{roos2002probabilistic,smailagic2002location},
the indoor area is divided into a grid and signal data is recorded from a set of fixed known points which is used as training data, generating a probability distribution of signal strengths given location values.  With new signal observations, the posterior distribution is computed and the location with the highest probability is chosen.  

The works of \cite{seshadri2005bayesian} and \cite{letchner2005large} both model user locations as states of a dynamic system with the noisy observations of RSSI data, and apply an implementation of Bayesian filters called particle filters to probabilistically estimate the system state (i.e., user location).  An initial training phase builds a wireless sensor map of the environment divided into cells by sampling at predefined points; this step is dependent on the assumed WiFi sensor model, either signal propagation or fingerprinting models.  Then location is estimated on a spatial connectivity graph; each motion update step of the Bayesian filter moves the user along an edge of this graph.  These probabilistic methods are powerful but suffers the drawback of potential high computational complexity inherent in the particle filters method whose worst-case complexity grows exponentially in dimensions of the state space.
Moreover, all these above-mentioned approaches all require an initial training phase, and therefore are not online and ad-hoc.

Ying's edit
