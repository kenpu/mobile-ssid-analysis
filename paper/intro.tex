\section{Introduction}

With the omnipresence of WiFi hotspots throughout the urban and suburban areas, it has become possible to perform location identification and
mobility analysis on any mobile device.  Given the nature of WiFi networks,
sensing the surrounding WiFi basic service set identifiers (BSSID) can be done
accurately with insignificant power requirement and latency, making it an ideal
way generate spatio-temporal mobility data streams.  The {\em raw} mobility data
stream consists of a time series of {\em readings}.  Each reading has a
timestamp, and a collection of triples $\left<\mathrm{bssid}, \mathrm{ssid},
\mathrm{strength}\right>$, where $\mathrm{bssid}$ is a universally unique
identifier of a WiFi hotspot, $\mathrm{ssid}$ is a user-defined name, and
finally $\mathrm{strength}$ is a numerical measure of the signal intensity of that particular WiFi hotspot.

% Distinguish our approach from others:

The problem of location inference based on WiFi scans has received a great deal
of attention (e.g., \cite{feng2012received,hatami2005comparative,
kushki2007kernel,liu2005signal,paul2008wi,
letchner2005large}).  Our work distinguishes from the existing
literature in several important ways.  We do not make any assumptions on the signal
propagation model, nor any prior knowledge of the ambient environment.  In the
{\em location identification phase}, our
algorithm continuously constructs a set of physical locations based on the
observation of the BSSID's and their signal strength.  Our emphasis is that
location identification is performed in an unsupervised, ad-hoc and incremental
fashion.  To make it suitable for the mobile runtime environment, the {\em
online} algorithm runs with sublinear time and space complexity.

% Outline the approach: mobility history, location, hierarchical organization

Our algorithm consists of a series of pipelined stages of WiFi reading
processing.  Each stage is a non-blocking online algorithm whose output is
consumed by the next stage.  

\noindent{\em Movements}

In the first stage, we detect {\em movements}
of the mobile device by maintaining a multiresolution timeline of the readings.  We
hierarchically partition the timeline to detect mobility at various time scales.
The multiresolution is used to reject false positives and cope with false
negatives in the WiFi readings.

\noindent{\em Physical Locations}

The segments generated by the movement detection phase are further processed to
build a database of {\em physical locations} based on the BSSID signatures of
the segments.  This allows us to map out the ambient environment over time.  We
note that movement detection and physical location identification are both
non-blocking, so that physical locations are identified instantly, and
incrementally refined.

\noindent{\em Semantic Clustering}

While the BSSID information provides accurate physical location identification,
they do not carry sufficient {\em semantic} value to the end user.  We do not use GPS for the sake
of minimizing power consumption.  It turns out that user-defined SSID names are a useful source of semantic knowledge of the physical locations.
Our algorithm analyzes the SSID names to organize the physical locations into a
semantically meaningful hierarchy.

\noindent{\em Location Inference}

Finally, the algorithm uses maximal likelihood estimation to infer locations
with semantic labels to the raw WiFi readings.


