\newcommand{\bssid}{\mathbf{B}}
\newcommand{\ssid}{\mathrm{BSSID}}

\section{Problem Definition}

A mobile device can make a scan.  We refer to each scan as a {\em reading}.
Each reading is defined as $\left<t(r), \bssid(r)\right>$ where $t(r)$ is
the timestamp of the reading, and $\bssid(r)\subseteq\mathrm{BSSID}$ is a set of
BSSID of the wifi hotspots that the scan detected. For each BSSID $b\in
\bssid(r)$ detected in the reading, we also have the SSID and the signal
strength, written respectively as:
$\ssid(b)$ and $s(b|r)$.  We assume that each BSSID has a unique SSID, while the
strength of a BSSID is specific to a given reading.

\begin{definition}
    A {\em timeline} $T$ is a sequence of readings.  
    We denote $T_i$ as the $i$-th reading of the timeline $T$.

    A {\em segment} of the timeline $S$ is a contiguous subsequence of $T$.
\end{definition}

Let $\mathcal{L}$ be a (unspecified) finite set of {\em locations}.

\begin{definition}[Location identification]
    A {\em location identification} problem consists of several subproblems:

    \begin{enumerate}
        \item {\em Identification} of the distinct locations $\mathcal{L}$ from
            a given timeline $T$.
        \item {\em Inference} of the location of a given reading.
    \end{enumerate}
\end{definition}
